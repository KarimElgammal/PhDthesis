\chapter{Graphene based sensors}
\label{grapheneSensors}

Here comes an overview of the sensory action of graphene with some insights on different factors behind the sensing mechanism. For a theoretical background for graphene and related properties, please refer to Chapter \ref{grapheneTheory}.
\section{Inauguration of graphene as a sensor}
Means of Atomic Force Microscopy experiments has ignited graphene's sensing behaviour visualising the first water adlayer on mica surface at ambient conditions~\cite{Xu2010} being a coating material on mica. Graphene's high selectivity to concentration changes for different ambient gases has enabled embedding it in sensory devices~\cite{Schedin2007, Yuan2013, Melios2017}. Graphene's sensitivity to gas molecules is mainly attributed to two factors: (1) graphene's $\pi$ orbitals~\cite{Neto2009} which interact with the adsorbates residing on top via van der Waals interactions, (2) graphene's high surface to volume ratio which is an advantage for all 2D materials. 

\section{2D materials as a gas sensor candidate}
By the millennium till now, vast number of materials have demonstrated gas sensing properties including low dimensional carbon-based materials as carbon nanotubes~\cite{Kong2000, Kuang2007}, graphene and its oxide~\cite{Ong2001, Schedin2007, Morozov2008,Fowler2009, Massera2011, Gautam2012, He2012, Dan2009, Ghosh2009, Ratinac2010, Sun2010, Yao2012, Basu2012, Liu2012, Gallouze2013, Nemade2013a, Borini2013, Yuan2013, Llobet2013, Saini2014, Zhang2013, Zhou2014, Hwang2014, Nakamura2015, Smith2015, Smith2016a, Smith2017, Xuge2017, Melios2017, Wu2015}. Recently, other 2D materials featuring a bandgap caught the attention as a promising candidate material for sensing devices. One example of such materials family is the transition metal di/tri-chalcogenides family (TMDCs/TMTCs)~\cite{Zhao2014}. For example, molybdenum disulfide has shown sensitivity to different gas molecules interacting on its monolayer~\cite{Yue2013, Zhao2014} showing sensitivity towards carbon monoxide/dioxide, ammonia, nitrogen monoxide/dioxide, methane, water molecules, nitrogen, oxygen and sulfur dioxide. Similarly, Dirac materials such as silicene have also demonstrated promising activity in the materials sensory behaviour~\cite{Feng2014}, showing sensitivity against a wide range of gases such as nitrogen monoxide/dioxide, sulfur dioxide, oxygen and ammonia, as well as formaldehyde~\cite{Wang2015}. Worth mentioning that Geim and Grigorieva' study on van der Waals heterostructures~\cite{Geim2013} has inaugurated the community to investigate further the resultant properties of stacking different 2D materials, revealing different properties which can have many applications, where sensors can be one of them~\cite{Mayor2016}. Interestingly, other low dimensional as metal oxides~\cite{Korotcenkov2007} or 1D materials as nanowires~\cite{Kong2000, Zhou2003, Wang2003} and tin oxide~\cite{Kuang2007} materials have proven sensing capabilities .
\section{Graphene as a gas sensor}

Graphene and other 2D materials have demonstrated sensing capabilities for an extensive collection of gases, both experimental and theoretical studies have focused on elucidating the different electronic properties within different ambient conditions emphasising different gases. The study in~\cite{Schedin2007} has experimentally enkindled the first graphene-based gas sensor achieving single molecule detection limit, in a way that the adsorbed gas molecules affect the charge carrier concentration and so the graphene device's resistance which is a direct measure of the device's sensitivity. Since then, studies focusing on various gases adsorption on graphene took place both theoretically and experimentally. Example of such gases are carbon dioxide and water molecules, where many of the studies examined their effect on graphene's electronic properties both theoretically~\cite{Ong2001, wehling2008doping, Wehling2008, Wehling2009, Ribeiro2008, Berashevich2009, Dai2009, Leenaerts2009, Yuan2010, Yang2011, Yoon2013, Freitas2011, Mishra2011, Voloshina2011, Deng2012, Paulla2013, Nemade2013a, Cazorla2013, Chen2014, Dutta2014, Xiao2014, Elgammal2017} and experimentally~\cite{Moser2008, Dan2009, Ghosh2009, Lu2009, Yavari2010, Yoon2011, Kalon2011, Yao2012, Yang2012, Tuan2013, Borini2013, Giusca2015, Tamilarasan2015, Smith2015, Hong2016, Smith2016a, Smith2016a, Melios2016, Panchal2016, Smith2017, Xuge2017, Melios2017}. 

Similarly, Graphene has showed sensitivity towards other gases such as carbon monoxide~\cite{Schedin2007, Ao2008, Zhang2009a}, oxygen~\cite{Yang2011, Yuan2013}, sulfur dioxide~\cite{Chen2014}, nitrogen monoxide/dioxide~\cite{Zhang2009dopants, Leenaerts2009, Dai2012, Panchal2016}, hydrogen sulfide~\cite{Yuan2010, Sharma2013} and ammonia~\cite{Zhang2009dopants, Leenaerts2009, Dai2009, Yuan2010}. Graphene sensing capabilities can be extended towards detecting complex bio-molecules~\cite{Lerner2017} such as DNA~\cite{He2010, Shao2010}, opening the capabilities for lab on chip applications for fast diagnosis or selectivity towards various bio-molecules~\cite{Barik2017}. This can enable graphene to enter the market of surface-based biosensors~\cite{Shao2010}. 
% %

Graphene's sensing behaviour towards adsorbates can differ according to graphene types~\cite{Smith2015}, thickness~\cite{Rafiee2012, Munz2015}, stacking orders~\cite{Melios2016}, defects~\cite{Banhart2011}, substrate effect~\cite{Hong2016, Wehling2008, Ashraf2016}. 
%
%
\section{Benchmarking against other materials based sensors}
Bench-marking graphene-based sensors against commercially available technologies are quite impressive. Honeywell\texttrademark developed a no expensive widely used humidity sensor based on polymer capacitive sensing mechanism~\cite{Delapierre1983}, with model code name (HIH-4000-001)~\cite{honeywell}. Its sensitivity can span the full relative humidity range yet achieves a response time of 10x and a recovery time of 40x making it pretty slower than graphene integrated CMOS resistive humidity sensor as demonstrated in~\cite{Smith2015}. The graphene-based humidity sensor does span 95\% relative humidity range, with a 5\% less than the commercial one.

Another example for well-developed humidity sensors in literature is the tin oxide~\cite{Kuang2007}, which is also resistive sensor and CMOS compatible. However, tin oxide based sensors do experience lower overall efficiency when compared to the graphene-based equivalent~\cite{Smith2015}, regarding the spanned relative humidity percentage range, response and recovery times and sensitivity to minute changes in humidity.
%
%
\section{Graphene's sensory action}
In the following subsections, we address different aspects that can play a role in graphene's sensing action towards different adsorbates.
%
%
\subsection{The nature of graphene-adsorbate interactions}
Environmental conditions and adsorbed molecules on top of graphene sheet do change the electronic properties of graphene regarding carrier concentration, resistance chance, work function and other properties~\cite{Kozbial2014, Panchal2016}. Pristine single-layered graphene is hydrophilic. However, it goes hydrophobic with stacked graphene layers~\cite{Munz2015} as per bernel stacked graphene (AB stacking). Moreover, the underlying substrate is proven to affect the hydrophilicity of the graphene sheet by doping mechanisms~\cite{Hong2016}. Adsorbates can dope the graphene sheet through p-doping~\cite{Panchal2016} as the adsorbates do attract electrons from graphene resulting in p-doping~\cite{Bollmann2015}. The sensing mechanism relies on changing the charge carrier concentration as well as charge carrier mobility~\cite{Melios2016, Melios2017tuning}.
%
%
\subsection{Effect of adsorbates on pristine graphene}
Adsorbates on top of pristine graphene have been investigated showing a charge transfer between the graphene sheet and the relaxed adsorbates on top. The charge transfer depends on the different orientations, geometries and relative positioning of the adsorbates~\cite{Leenaerts2008, Leenaerts2009}. For adsorbates of water type: the charge transport depends on the water molecule orientation, in which the charge transport is from the water molecule to the graphene sheet when the water's oxygen is the closest to the graphene sheet~\cite{Freitas2011} and reversed when the hydrogen atom is the closest~\cite{Leenaerts2008}. Adsorbates of water or ammonia existing on top of either single layered or bi-layered graphene sheets can, in some cases, open a bandgap opening in order of few tens of meV~\cite{Ribeiro2008}. The adsorbates orientations can depend on the graphene sheet charge where the hydroxylic bond within the water molecule does point towards the graphene sheet in case of negatively charged graphene~\cite{Tuan2013} and vice versa for positively charged graphene. Large concentrations of water adsorbates (forming icelike structures) on top of a pristine graphene sheet can result in comparably large net dipole moment accumulation, in which has a net doping effect on the graphene sheet leading to changing the electronic charges around the graphene sheet~\cite{Leenaerts2009}.

All in, the presence of water adsorbates concentration on both sides of the graphene sheet as well as its relative orientation either pointing towards the graphene sheet or opposite do have a resultant effective doping mechanism which changes the charge transfer to and from the graphene sheet~\cite{Leenaerts2009, Freitas2011, Tuan2013}. However, the change in the electronic structure is not dramatic when adsorbates are present~\cite{Wehling2008} and is quite minute as long as the study is concerned with the effect of adsorbates on pristine graphene. 
%
%
\subsection{Effect of adsorbates on defected graphene}
Defects can take place within the graphene sheet itself where common defects within the graphene sheet can be either categorised into point defects and 1D defects~\cite{Banhart2011}. Where point defects can involve Stone-Wales (SW) defect~\cite{Stone1986}, the typical single vacancy (SV)~\cite{Ugeda2010}, double and multiple vacancies, carbon adatoms, embedded foreign adatoms, substitutional impurities (introducing dopant atoms), defective topology and so on. While 1D defects can involve line defects, edge defects, and similar defects that result from separated domains within the graphene sheet characterised by different lattice orientations~\cite{Banhart2011}. Defects types can involve having unusual buckled or rippled graphene sites~\cite{Dutta2014}.

Graphene-based sensors featuring vacancy defects~\cite{Lee2016} has achieved sensitivity enhancement for various adsorbates signalling 33\% improvement for adsorbates of nitrogen dioxide and 614\% improvement for ammonia while compared to pristine graphene. Moreover, line defects can have a remarkable influence on graphene's electronic structure and hence the sensitivity towards adsorbates on top~\cite{Souza2018}. 

Experimental chemical and physical defects can alter the humidity sensitivity of graphene surfaces~\cite{SON2017defect} where the chemical defects (obtained by reactive ion etching) do have a more substantial effect on the sensitivity than the physical defects (via PMMA coating). 
%
%
\subsection{Effect of adsorbates on doped-graphene}
Graphene sheets sensing properties can depend on dopants existence within. Doping the graphene sheet itself changes the electronic and sensing properties~\cite{Zhang2009dopants, Panchal2016}, p-doping can be achieved via boron and nitrogen~\cite{Yuan2010, Deng2011, Deng2012}, gallium, germanium, arsenic and selenium dopants~\cite{Chen2014, Denis2014}, silicon doping~\cite{Zou2011}, aluminium doping~\cite{Dai2009, Sharma2013}. For example, aluminium-doped graphene has proven different electronic structure properties when adsorbing hydrogen fluoride molecules compared to pristine graphene~\cite{Sun2010}. Moreover, adsorption of hydrogen fluoride on top of Aluminium doped graphene has a chemisorption nature while it is physisorption for the pristine graphene case~\cite{Sun2010}. 

As doping graphene can alter the adsorption nature of adsorbates on top of a graphene sheet: an extensive study focusing on the adsorption nature of molecular hydrogen on top of graphene~\cite{Gallouze2013} has revealed that the adsorption can either be physisorption or chemisorption. Depends on graphene's dopant type: it is physisorption when the dopants are boron, iron, cobalt and nitrogen. While it is chemisorption when the dopants are hydrogen, beryllium, oxygen, sodium, aluminium, silicon, calcium, titanium, vanadium, chromium, nickel, copper and lithium.
%
%
%
%
\subsection{Effect of adsorbates on stacked graphene}
Different stacking orders in graphene do alter the carrier concentration and work function, while single-layered graphene is the most sensitive to different ambient conditions~\cite{Giusca2015, Panchal2016}. Adding only one layer resulting in bilayer graphene can decrease the sensitivity. Within bilayered-graphene, the bottom graphene layer is affected by charges coming from the substrate, while the top layer is affected by the adsorbates on top~\cite{Xuge2017}. The doping type, in this case, is also of acceptor type (p-doping)~\cite{Melios2016}.
%
%
%
%
\subsection{Effect of adsorbates on graphene with the influence of substrate}
Defects modifying graphene's electronic properties can extend towards the substrate surface defects in which the graphene sheet is residing on top~\cite{Wehling2008, Wehling2009, Elgammal2017}. Such commonly found substrate surface defects do contribute by inducing a net dipole moment with the presence of adsorbates on top of the graphene sheet. Such dipole moment accounts for a doping effect which results in changing the graphene's electronic structure and hence can alter the sensitivity~\cite{Wehling2008, Smith2015, Smith2017, Melios2017, Elgammal2017}. 

For example, adsorbates of oxygen with the presence of silica substrates dope the graphene sheet due to the couplings to the graphene sheet and the coupling between the graphene and the substrate~\cite{Ryu2010}. Oxygen molecules doping effect is of acceptor type (hole doping)~\cite{Liu2008}. Typically, such p-doping of graphene takes place when graphene exposes to regular atmospheric conditioning, i.e. exposure to water, carbon dioxide, oxygen and other ambient molecules in the air~\cite{Levesque2011, Anton2012}. Graphene p-doping action is not only due to the adsorbate in ambient conditions~\cite{Melios2017tuning} but also charges arising from the underlying silica substrate can induce such p-type doping of graphene~\cite{Melios2016}. Electronically, the underlying substrate surface defects can facilitate the doping effect by shifting the Dirac point by 0.5 eV as proven in~\cite{Levesque2011}. 

Moreover, altering the degree of hydrophobicity~\cite{Chen2014, Belyaeva2017} can directly affect the p-doping in graphene. For example, applying an electric field to the graphene-substrate system can alter the degree of hydrophobicity, resulting in a maximisation of the substrate induced doping~\cite{Hong2016}. The applied electric field shifts the Fermi level relative to the Dirac point changing the graphene doping from n-type to p-type doping.  
%
%
\section{Summary}
We have demonstrated a short overview of the influence of several parameters on graphene's sensing action with a big emphasis on the effects coming from the underlying substrate. As graphene has proven sensitivity to ambient conditions, we should give a careful treatment when constructing graphene-based sensory devices considering all the discussed parameters.
