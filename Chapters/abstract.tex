\begin{abstract}
Nowadays, electronic devices span a diverse pool of applications, especially when getting smaller and smaller satisfying the \textit{more than Moore} paradigm. To further develop this, studies focusing on material design toward electronic devices are crucial. Accordingly, we present a theoretical study investigating the possibility of graphene as a promising material for such electronic devices design. We focus on graphene and graphene-based sensors. Graphene is known to have outstanding electronic and mechanical properties making it a game changer in the electronic design in the so-called 'post-silicon' industry. It is stronger than steel yet the thinnest material ever known while overstepping copper regarding electronic conductivity.

In this thesis, we perform first-principle \textit{ab-initio} density functional theory (DFT) calculations of graphene in different sensing ambient conditions, which allows fast, accurate and efficient investigations of the electronic structure properties. Principally, we centre our attention on the arising interactions between the adsorbates on top of the graphene sheet and the underlying substrates' surface defects. The combined effect of the impurity bands arising from these defects and the adsorbates reveals a doping influence within the graphene sheet. This doping behaviour is responsible for different equilibrium distances and binding energies for different adsorbate types as well as substrates. Moreover, we briefly investigate the same effect on double layered graphene under the same ambient conditions. 

We extend the studies to involve various types of substrates with different surface conditions and different adhesion nature to graphene. We take into consideration the governing van der Waals interactions in describing the electronic structure properties taking place at the graphene sheet interfacing both with the substrates below and the adsorbates above. Furthermore, we investigate the possibility of passivating such action of graphene sensing towards adsorbates to inhibit the graphene's sensing action as devices passivation becomes a necessity for the ultimate purpose of achieving \textit{more than Moore} applications. Which in turn result in the optimal integration of graphene-based devices with different other devices functionalities on the same resultant chip.

In summary, graphene, by means of first-principle calculations verification, shows a promising behaviour in the sensor functionality enabling \textit{more than Moore} applications for further advances. \\ \\
\textbf{Keywords: graphene, \textit{ab-initio}, humidity, carbon dioxide, substrate, DFT, vdW, first-principle.}
\end{abstract}
\endinput