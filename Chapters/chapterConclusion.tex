\chapter{Conclusions and future outlook} 
In this work, we discussed a theoretical investigation of a possibly \textit{more than Moore} application of graphene's sensitivity towards humidity and carbon dioxide molecules. 

We focused on extracting the electronic structure properties of single-layered pristine graphene as a sensor, more especially as humidity and carbon dioxide sensors. Similarly, we extended the studies towards investigating the double layered graphene sensing action towards the same adsorbates. We demonstrated the effect of passivation on such sensing action resulting in dysfunctioning sensors for better integrity of high valued systems in the \textit{more than Moore} paradigm. All in all, we consider this approach crucial for optimal delivery of high-valued systems combining system-on-chip (SoC) with system-in-package (SiP) as direct usability and applicability of \textit{more than Moore}.

In particular, we performed the studies on graphene within different ambient conditions through first-principle \textit{ab-initio} calculations. We focused on the combined effect of the underlying substrates and adsorbates on top of the graphene sheet itself. This effect is attributed to be a doping effect which is responsible for different adhesive forces added to the graphene's interface with the substrate as well as to the adsorbates on top. These results are in line with related literature available in~\cite{Wehling2008, Liu2008, Levesque2011, Anton2012, Hong2016, Ashraf2016, Melios2017}. Those interactions are weak and governed mainly by van der Waals forces. Thus, the study also included different investigations of such weak interaction of graphene on different surfaces with metal, semiconductor and insulator nature where physisorption or chemisorption adhesive nature govern graphene's adhesion with those substrates. 

In summary, we demonstrated and highlighted some factors that affect the sensing behaviour of pristine graphene where the electronic structure of such graphene-adsorbates interactions relies on the stacking orders as well as the underlying substrate-induced doping effect due to common defects available on the substrate surfaces. 

As future work, expanding the study to involve other molecules' combined effect with the substrates via accurate van der Waals methods is an important step towards incorporated into more sensing devices. Examining other 2D materials can have potential interest opening more applications for gas sensory paradigm~\cite{Yue2013, Zhao2014}, especially if they feature either direct or indirect bandgaps. Moreover, studying the conductance employing transport calculations within the non-equilibrium Green's functions scheme~\cite{DattaAtomstotransistors} is crucial for ultimate benchmarking purposes and to explore the transport properties of such sensing devices from a purely theoretical point of view. Finally, merging results from multi-scale simulation paradigm and \textit{ab-initio} can be useful for understanding more properties of such systems in-design~\cite{fiori}.
\endinput