\selectlanguage{swedish}
\begin{abstract}
%
Elektroniska komponenter används i allt vidare utsträckning, och deras användning ökar i takt med att de blir mindre och mindre samtidigt som deras prestanda ökar, enligt det paradigm som brukar kallas ''more than Moore''. För att att göra ytterligare framsteg i denna riktning är grundläggande studier som fokuserar på materialdesign och tillverkning av nya typer av elektroniska komponenter avgörande. I den här avhandlingen presenteras teoretiska studier av grafen-baserade komponenter. Grafen är ett mycket intressant material för framtidens elektroniska komponenter. Specifikt fokuserar vi på grafenbaserade gas-sensorer. Grafen är känt för att ha mycket ovanliga elektroniska och mekaniska egenskaper som gör det till ett unikt material för "post-silicon"-design av elektronik. Det är starkare än stål och är samtidigt världens  tunnaste material. Samtidigt har det bättre elektrisk ledningsförmåga än koppar.

Täthetsfunktionalsteori (DFT) har använts för att beräkna hur den elektroniska strukturen hos grafen ändras som funktion av substratmaterial och typ av molekyler som adsorberats på grafenets yta. DFT är en beräkningsmetod som medger simuleringar med hög precision samtidigt som den är relativt snabb. I studierna har DFT kombinerats med olika modeller för van der Waals-interaktionen.
En viktig aspekt i de studier vi presenterar här är interaktionen mellan adsorbat-molekylerna ovanpå grafenet och ytdefekterna hos det underliggande substratet. De orenhetsband som härrör från defekterna, i kombination med adsorbat-molekylerna, skapar en slags dopningseffekt som ändrar elektronstrukturen hos grafenet. Därmed kan även de elektriska transportegenskaperna ändras hos grafenet, vilket möjliggör elektrisk detektion av molekylerna.

Vi har även studerat sensorer byggda med dubbelskiktad grafen. Dessutom har vi gjort en systematisk studie av hur grafen binder till ett stort antal substrat samt även hur man kan passivisera grafen så att den elektriska ledningsförmågan inte ändras vid molekyladsorption. Detta sista är viktigt för "more than Moore"-tillmämpningar, där ett centralt designkriterium är att kunna integrera många funktioner på samma chip.
%
\end{abstract}
\endinput
