\chapter{Method and calculational details}
\label{chapter:calcDetails}
Here comes an overview of the used DFT code, pseudopotentials, cutoffs and other details related to the calculations setup.
\section{Used code}
All the numerical simulations throughout this thesis' manuscripts have been performed using density functional theory (DFT)~\cite{Hohenberg1964, Kohn1965}. The Kohn-Sham equation described in section \ref{KS} has been solved within the formalism of the planewave basis sets and pseudopotentials~\cite{Vanderbilt1990}. We have carried out the calculations using the integrated open-source software distribution suite of Quantum ESPRESSO~\cite{Giannozzi2009} abbreviated (QE) which is an integrated suite of electronic structure calculations codes based on DFT using planewave basis sets and pseudopotentials. It can be downloaded from~\cite{QE_link} for free.
\section{Used pseudopotentials}
Pseudopotential choices vary throughout this thesis' manuscripts. Here comes a brief overview of the chosen ones: 
\begin{itemize}
    \item \textbf{The HSCV pseudopotentials library}~\cite{PP_HSCV}: It is norm-conserving pseudo-potentials. It stands for Hamann, Schluter, Chiang and Vanderbilt (HSCV). We have used it at a tested cutoff of 130 Ry. We used it in the manuscripts representing Papers \hyperref[P1]{$\Rmnum{1}$}, \hyperref[P4]{$\Rmnum{4}$}, \hyperref[P5]{$\Rmnum{5}$} with a planewave energy cutoff value of 130 Ry.
\item \textbf{The GBRV pseudopotentials library}~\cite{Garrity2014}: it is ultrasoft pseudopotentials downloadable from Rutgers database~\cite{GBRVdatabase}. It is highly accurate and computationally inexpensive open-source pseudopotential library designed and optimised for use in high-throughput DFT calculations. It has relatively low recommended cutoffs of 40 Ry for the planewave and 200 Ry for the charge density. It has been used in the manuscripts representing Papers \hyperref[P2]{$\Rmnum{2}$}, \hyperref[P7]{$\Rmnum{7}$} with the recommended cutoffs.
\item \textbf{The SSSP pseudopotentials library}\cite{SSSP}: It is an ambitious pseudopotentials curation effort done on most of the commonly available pseudopotential libraries for the QE suite leading to the identification of optimal pseudopotentials. In which, they are classified into either efficient (relatively faster) or accurate choices of pseudopotential sub-libraries. Energy cutoff values, theorised in section \ref{planewaves}, are provided for almost every element in the periodic tables, where both cutoffs for the wavefunction and charge-density were chosen according to convergence tests concerning cohesive energies, stresses and phonon frequencies performed for each element~\cite{SSSP}. The SSSP library was originally constructed within the framework of AiiDA~\cite{Pizzi2016, AiiDA_link}, an open-source platform to manage and automate scientific computational work-flows without human intervention. The SSSP pseudopotentials were exhaustively tested and chosen from various sources~\cite{Garrity2014, Kucukbenli2014, Dalcorso2014, Schlipf2015, Willand2013, Topsakal2014}. The accuracy sub-library has been bench-marked providing -to date- the best overall agreement with the all-electron codes results~\cite{Lejaeghere2014, Lejaeghere2016}. This SSSP accuracy library has been used in the manuscript representing Paper \hyperref[P3]{$\Rmnum{3}$} with the assigned recommended cutoffs.
\item \textbf{ONCV pseudopotentials library}~\cite{Schlipf2015}: It is norm-conserving scalar relativistic pseudopotentials, it stands for Optimized Norm-Conserving Vanderbilt. It is considered as the updated version of the HSCV with an efficient lower cutoff, 60 Ry, about half the value needed for HSCV regarding the same systems. This pseudopotential was the choice in the manuscript representing Paper \hyperref[P6]{$\Rmnum{6}$}, where it has been used at the recommended cutoff of 60 Ry.
\end{itemize}
\section{Used functionals}
As we have pointed before in the subsection \ref{nonlocalvdW} for the importance of modelling vdW interactions, we used either non-local vdW functionals~\cite{Berland2014vdWxc} or semi-empirical corrections~\cite{Grimme2004, Grimme2006} to account for the dispersive interactions taking place within the slabs, graphene sheets and the available adsorbates on top.
\section{Systems modelling}
All systems under study within all the manuscripts involved slabs with incorporated vacuum to avoid any interaction within periodic images constructed by the code. In the manuscripts systems, we treated both slab surfaces symmetrically where the termination end of the top and bottom of the slab are identical. That serves the purpose of cancelling any resultant intrinsic dipole moment resulting from the slabs in case they are treated asymmetrically. We built the slabs to be quite thick resulting in bulk-like middle layers inside the slabs. We followed another approach in the manuscript representing Paper \hyperref[P3]{$\Rmnum{3}$}, which is the dipole correction~\cite{Bengtsson1999} method implemented in the QE package. Such correction can compensate for the artificially produced field within the supercell's vacuum by imposing a saw-like potential cancelling the artificially generated one taking place in the vacuum.
\section{Other calculational details}
\begin{itemize}
    \item Within most of the manuscripts in this thesis, we have generated QE input files from bulk Crystallographic Information File (CIF) files via the CiF2CeLL utility code~\cite{Torbjorn2011, CIF2CELLlink}. We fetched most of those CIFs from the materials project database~\cite{Anubhav2013, MaterialsProjectLink}. Those CIFs are available after undergoing satisfactory relaxations processes through their automated machinery available at the Materials Project repositories. Materials project uses the ICSD library~\cite{Bergerhoff1983, ICSDLink} as a source of the structures being fed into their computational workflows.
    \item We calculated the charge transfer using the Bader charge analysis~\cite{Bader1994} via a utility code computationally implemented in~\cite{Henkelman2006, Tang2009}. 
    \item We computed Löwdin charge analysis~\cite{Lowdin1955} as implemented in QE package.
    \item We used the visualisation tools xCrysden~\cite{Kokalj1999} and VESTA~\cite{Momma2011} for visualising system components and charge density differences. Both utilities can be downloadable from~\cite{xCrysdenLink} and~\cite{VESTALink}
\end{itemize}
\endinput